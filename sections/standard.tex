\section{Normalverteilung}
Zentraler Grenzwertsatz: \begin{math}
    f(x) = \frac{1}{\sigma * \sqrt{2 * \pi}} * e^{-\frac{1}{2} * \left(\frac{x - \mu}{\sigma}\right)^2}
\end{math} dieser ist relevant.

Mathcad Befehle:
\begin{table}[h]
    \caption{Mathcad Befehle}
    \begin{tabularx}{\textwidth}{|l|X|}
        \hline
        \textbf{Befehl}            & \textbf{Beschreibung}                                                                        \\ \hline
        dnorm (x, $\mu$, $\sigma$) & Wahrscheinlichkeitsdichte f\"ur $X$ = $x$. Gau\ss'sche Glockenkurve                          \\ \hline
        pnorm (x $\mu$, $\sigma$)  & Wahrscheinlichkeitsverteilung f\"uer den Schwellenwert $x$. Wahrscheinlichkeit $P(X \leq x)$ \\ \hline
        qnorm (p, $\mu$, $\sigma$) & Inverse Wahrscheinlichkeitsverteilung $P(X \leq x) = p$ nach x aufl\"osen.                   \\ \hline
    \end{tabularx}
\end{table}

Eigenschaften der Glockenkurve
\begin{itemize}
    \item Beim Erwartungswert $\mu$ besitzt die Glockenkurve ihr maximum.
    \item Die Standardabweichung $\sigma$ bestimmt die Breite
          \begin{itemize}
              \item $\sigma$ ist die Wurzel aus der Varianz
              \item $\sigma$ ist die Wurzel aus dem Erwartungswert der Quadrate
          \end{itemize}
    \item Die Fl\"ache unter der Kurve ist immer 1
    \item $P(X \le x_{0}) = P(X \leq x_{0}) = \int_{-\infty}^{x_{0}} f(x) dx = F(x_{0})$
\end{itemize}

\begin{tikzpicture}
    \begin{axis}[
            axis lines=left ,
            xlabel=$x$,
            ylabel=$f(x)$,
        ]
        \addplot[domain=-3:3, samples=100, color=blue, mark=none] {1/sqrt(2*pi)*exp(-1/2*(x)^2)};
    \end{axis}
\end{tikzpicture}

\begin{math}
    P (X \leq x_{0}) = P (X \le x_{0}) = \begin{cases}
        \int_{\infty}^{x_{0}} f(x) dx \\
        1 - P(X \leq x_{0}) = 1 - F(x_{0})
    \end{cases}
\end{math}

\textbf{Bsp.} Abf\"ullanlage f\"ur \"Olkanier X \ldots Abf\"ullanlage in Liter $\mu$ =
5,00 Liter $\sigma$ = 0,09 Liter
\begin{enumerate}[(a)]
    \item Wahrscheinlichkeit das F\"ullmenge h\"ochstens 5,10l betr\"agt? \newline $P ( X
              \leq 5,1) = F(5,1) = pnorm(5,1 ; 5 ; 0.09) = 86.7\%$
\end{enumerate}